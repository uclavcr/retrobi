\chapter{Úvod}

\section{Cíl projektu}

Tento dokument vznikl jako implementační dokumentace projektu digitalizace a online zpřístupnění kartotéky retrospektivní bibliografie české literární vědy. Projekt má za cíl především usnadnit vyhledávání a listování existující kartotékou bez nutnosti fyzického přístupu k lístkům. Využitím moderních technologií je možné práci s těmito lístky umožnit všem, kteří online katalog navštíví přes internet. Významným přínosem a součástí projektu je i provedení OCR rekognoskace při importu dat a možnost fulltextového vyhledávání v těchto datech. Shromážděné lístky bude dále možné v průběhu času spravovat a informace na nich obsažené stále přesněji strukturovat.

Projekt je výsledkem projektu \uv{Digitalizace lístkového katalogu Retrospektivní bibliografie české literární vědy 1775--1945} (VZ09004), který v rámci programu INFOZ v letech 2009--2011 laskavě finančně podpořilo Ministerstvo školství, mládeže a tělovýchovy České republiky.

Vlastní projekt je možné rozdělit do několika částí:

\begin{itemize}
\item{návrh procesu plnění a údržby katalogu (částečně jde mimo tuto dokumentaci, protože je v režii zaměstnanců ÚČL);}
\item{vytvoření jednoúčelových nástrojů pro zpracování a import obrázků do databáze tak, aby odpovídaly navrženému procesu;}
\item{vytvoření webové aplikace určené k procházení a správě nashromážděných dat.}
\end{itemize}

Poměrně přesné zadání projektu (včetně všech původních požadavků ze strany ÚČL) lze najít v zadávací dokumentaci. V průběhu času se tyto požadavky samozřejmě lehce měnily, usměrňovaly a stabilizovaly, jak se projekt vyvíjel.

\section{Dostupnost}

Zdrojové kódy aplikace byly předány na datových nosičích a jejich záloha je umístěna na verzovacím systému SVN společnosti inSophy~s.r.o. Veškerý software od třetích stran potřebný k sestavení a provozování systému je šířen zdarma či pod svobodnou licencí. Celý proces kompilace včetně stažení potřebných závislostí a knihoven je řízen systémem Maven a je popsán v následujících kapitolách.
